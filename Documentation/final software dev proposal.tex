

%----------------------------------------------------------------------------------------
%	PACKAGES AND OTHER DOCUMENT CONFIGURATIONS 
%----------------------------------------------------------------------------------------

\documentclass[paper=a4, fontsize=11pt]{scrartcl} % A4 paper and 11pt font size

\usepackage[T1]{fontenc} % Use 8-bit encoding that has 256 glyphs
\usepackage{fourier} % Use the Adobe Utopia font for the document - comment this line to return to the LaTeX default
\usepackage[english]{babel} % English language/hyphenation
\usepackage{amsmath,amsfonts,amsthm} % Math packages

\usepackage{lipsum} % Used for inserting dummy 'Lorem ipsum' text into the template

\usepackage{sectsty} % Allows customizing section commands
\allsectionsfont{\centering \normalfont\scshape} % Make all sections centered, the default font and small caps

\usepackage{fancyhdr} % Custom headers and footers
\pagestyle{fancyplain} % Makes all pages in the document conform to the custom headers and footers
\fancyhead{} % No page header - if you want one, create it in the same way as the footers below
\fancyfoot[L]{} % Empty left footer
\fancyfoot[C]{} % Empty center footer
\fancyfoot[R]{\thepage} % Page numbering for right footer
\renewcommand{\headrulewidth}{0pt} % Remove header underlines
\renewcommand{\footrulewidth}{0pt} % Remove footer underlines
\setlength{\headheight}{13.6pt} % Customize the height of the header

\numberwithin{equation}{section} % Number equations within sections (i.e. 1.1, 1.2, 2.1, 2.2 instead of 1, 2, 3, 4)
\numberwithin{figure}{section} % Number figures within sections (i.e. 1.1, 1.2, 2.1, 2.2 instead of 1, 2, 3, 4)
\numberwithin{table}{section} % Number tables within sections (i.e. 1.1, 1.2, 2.1, 2.2 instead of 1, 2, 3, 4)

\setlength\parindent{0pt} % Removes all indentation from paragraphs - comment this line for an assignment with lots of text

%----------------------------------------------------------------------------------------
%	TITLE SECTION
%----------------------------------------------------------------------------------------

\newcommand{\horrule}[1]{\rule{\linewidth}{#1}} % Create horizontal rule command with 1 argument of height

\title{	
\normalfont \normalsize 
\textsc{university, school or department name} \\ [25pt] % Your university, school and/or department name(s)
\horrule{0.5pt} \\[0.4cm] % Thin top horizontal rule
\huge Project 13 -- Feet of Delivery and pick-up Vehicle management System \\ % The assignment title
\horrule{2pt} \\[0.5cm] % Thick bottom horizontal rule
}

\author{Sheena Philip,Linda Khumalo,Kessigan Subramanium,Phumzile Dhlwathi} % Your name

\date{\normalsize\today} % Today's date or a custom date

\begin{document}

\maketitle % Print the title

%----------------------------------------------------------------------------------------
%	PROBLEM 1
%----------------------------------------------------------------------------------------

\section{	The environment}

\begin{table}[ht!]
\centering
\caption{Programming resources}
\label{my-label}
\begin{tabular}{|l|l|}
\hline
Programming languages                                     & Python,JavaScript                                                         \\ \hline
Editors                                                   & gedit                                                          \\ \hline
IDE                                                       & Eclipse                                                        \\ \hline
Databases                                                 & PostgreSQL                                                     \\ \hline
UI Frameworks                                             & \begin{tabular}[c]{@{}l@{}}Bootstrap,\\ CSS, HTML\end{tabular} \\ \hline
Documentation                                             & TexStudio                                                      \\ \hline
\begin{tabular}[c]{@{}l@{}}Version\\ Control\end{tabular} & Git/Github                                                     \\ \hline
\end{tabular}
\end{table}

%\begin{align} 
%\begin{split}
%(x+y)^3 	&= (x+y)^2(x+y)\\
%&=(x^2+2xy+y^2)(x+y)\\
%&=(x^3+2x^2y+xy^2) + (x^2y+2xy^2+y^3)\\
%&=x^3+3x^2y+3xy^2+y^3
%\end{split}					
%\end{align}



%------------------------------------------------

\section{	Requirements Analysis}

\begin{itemize}
	\item Make a website for managing delivery trucks: 
		\begin{itemize}
		 
			\begin{itemize}
			\item The website should serve as a management and tracking system for the fleet of vehicles
			\item There are two main types of users, the dispatcher and a driver
			\item Both types of users will use a desktop version of the website
			\item  The dispatcher should be able to see all available trucks and all non-available trucks and should enter the destination, type of goods, duration of trip and ultimately assign a driver to a destination
			\item Algorithms for determining the driver, and other variables on the system should be researched/developed and implemented
			\end{itemize}
		 
		\end{itemize}
		\item	Find project data or make dummy data on fleet of delivery vehicles
	\item	Design the back end structure and front end interface
		\item	Implement the design 
		\item	Unit test the system as progress is being made
	 
\end{itemize}





%------------------------------------------------

\subsection{	BACK--END : Linda, Phumzile}


\begin{itemize}
	\item  make a database and populate it
	\item  implement algorithms which can calculate things related to the front end, such as:
		\begin{itemize}
		 \item scheduling
		\item capacity
		\item goods transferred/pickup 
		\end{itemize}
	\item set up a server
	\item link database to server
	\item	link front end to server
	\item	write unit tests  
\end{itemize}




\subsection{	FRONT END : Kessigan, Sheena}

\begin{itemize}
	\item  Make user interfaces for the different users

		\begin{itemize}
		 \item Choose a bootstrap template and make minor modifications
		\item Plot the views and visualisations using JavaScript libraries such as d3, google charts, am charts on the template Users: For each user there is a User Interface based on the design:
		
		\begin{itemize}
		\item Program
		\item scheduling
		\item Dispatcher (inputs the variables so the algorithms can run and return a result)
		\item capacity
					\item goods transferred/pickup
					\item location
				\item Truck driver
					\item schedule
					\item goods transferred/pickup
					\item location
		\end{itemize}
		
		\end{itemize}
	\item Write unit tests 
\end{itemize}





%\paragraph{Heading on level 4 (paragraph)}
%
%\lipsum[6] % Dummy text
%
%%----------------------------------------------------------------------------------------
%%	PROBLEM 2
%%----------------------------------------------------------------------------------------
%
%\section{Lists}
%
%%------------------------------------------------
%
%\subsection{Example of list (3*itemize)}
%\begin{itemize}
%	\item First item in a list 
%		\begin{itemize}
%		\item First item in a list 
%			\begin{itemize}
%			\item First item in a list 
%			\item Second item in a list 
%			\end{itemize}
%		\item Second item in a list 
%		\end{itemize}
%	\item Second item in a list 
%\end{itemize}
%
%%------------------------------------------------
%
%\subsection{Example of list (enumerate)}
%\begin{enumerate}
%\item First item in a list 
%\item Second item in a list 
%\item Third item in a list
%\end{enumerate}
%
%%----------------------------------------------------------------------------------------

\end{document}